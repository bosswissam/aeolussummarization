\chapter{Auditing Using Summarization}

In this chapter give examples of how our system can be used to summarize and access information for the Mint application, as well as mark certain events.

\section{Summarizing User Sessions}
\label{sum:sessions}
In this example we show how to summarize a user's actions per session. In particular, for a session for the user Bob, we will record which of the user actions presented in section \ref{mint:impl} (e.g. addBank, downloadTransactions) he requested.

This should not be a hard task since we use application-level logging to track all of Bob's actions. However, we will need to create some context before producing the summaries to identify any one of Bob's sessions.

First, we initialize the query system, which we will use to add and run queries:

\begin{lstlisting}[language=Java]
//QuerySystem uses the singleton paradigm
QuerySystem system = QuerySystem.getInstance();
\end{lstlisting}

We then set up a query that retrieves the princpal ID and data tag for a particular user. Note that in our implementation, the name of the LOG table in the database is \emph{events}.

\begin{lstlisting}[language=Java]
String users_query = ``select to_array(app_args)[1] as pid, to_array(app_args)[2] as tag, event_counter from events where app_op='mint-user-signup' and to_array(app_args)[0] = ?'';

system.addQuery(``user-info'', users_query);
\end{lstlisting}

\noindent
This query is based on logging events when users sign up as described in section \ref{audit:mint-example}. The \emph{app\_args} attribute contains three pieces of information; the username, principal ID, and data tag, as a comma-delimited string.

This query produces a tuple with the principal ID in the first column, data tag in the second column and event\_counter of the signup event in the third column for a particular user. It does so by looking for signup events, which have the attribute \emph{app\_op} as ``mint-user-signup'', for a particular username, stored in the first index of the \emph{app\_args} attribute. 

We now add some queries that will help identify the logins and logouts for a particular user:

\begin{lstlisting}[language=Java]

/**
 * When a user, say Bob,  logs in, we execute the following code:
 * AeolusLib.createEvent(``mint-user-login'', Arrays.asList(new String[]{``Bob''}));
 * and similarly for logouts.
*/

// The following query produces a tuple that contains the login
// times and event_counters for a particular user.
String user_logins = ``select timestamp as start, event_counter from events where app_op='mint-user-login' and app_arg[0]=?'';

// The following query produces a tuple that contains the logout
// times and event_counters for a particular user.
String user_logouts = ``select timestamp as start, event_counter from events where app_op='mint-user-logout' and to_array(app_args)[0]=?'';

// add the queries to the QUERY table
system.addQuery(``logins'', user_logins);
system.addQuery(``logouts'', user_logouts);
\end{lstlisting}

\noindent
We assume that there is a one-to-one pairing between user login events and user logout events. Users cannot login if they are already logged in, and they are logged out after a certain period of time. Hence, for events far enough in the past, this is a fair assumption to make. We can now use the information from those queries to identify Bob's sessions:

\begin{lstlisting}[language=Java]
// A ResultSet is a set of records.
ResultSet bob_logins = system.runQuery(``logins'', new Object[]{``Bob''});
ResultSet bob_logouts = system.runQuery(``logouts'', new Object[]{``Bob''});

// The first login and logout for Bob will define the endpoints of the
// first session.

if((!bob_logins.next()) || (!bob_logouts.next())){
  return;
}

// Information for the first session
String username = ``Bob'';
long start_counter = bob_logins.getLong(3);
long end_counter = bob_logouts.getLong(3);
Timestamp start_time = bob_logins.getTimestamp(2);
Timestamp end_time = bob_logouts.getTimestamp(2);
\end{lstlisting}

\noindent
A ResultSet is a set of records. Calling the \emph{next()} method on the ResultSet objects move their cursor to the first record. We first make sure that Bob logged in to our system once in the past and then logged out. We then retrieve information from the ResultSet using the \lstinline$getLong()$ and \lstinline$getTimestamp()$. These methods retrieve a value at the given column number from the ResultSet and cast those values to their relevant types. Details of the ResultSet class and how its methods work can be found in \cite{jdoc}.

Table \ref{table:user-sessions} shows a possible table that contains a list of user sessions. We are now left with the problem of identifying the different actions that took place within any sessoin (identifiable by a start and end counters).

\begin{table}
\begin{verbatim}
 username | start_counter | end_counter 
----------+---------------+-------------
 wissam   |           168 |         373
 david    |           482 |        1075
 dan      |           830 |        1086
 Ted      |          2644 |       22675
 Leah     |          2690 |       19437
 Alex     |          2736 |        7786
 Jamie    |          2782 |        5991
 Dina     |          2828 |        7336
 Mark     |          2874 |       18576
 Helen    |          2920 |        6312
 Jon      |          2966 |       18370
 Mike     |          3072 |       18321
 Jack     |          3118 |       15342
 Eve      |          3164 |        7735
 Charlie  |          3210 |        7434
 Phil     |          3450 |       19535
 Tod      |          3684 |        5895
 Chris    |          3924 |       18075
 James    |          3970 |        7635
 Dave     |          4110 |       18223
 Aline    |          4253 |       18773
 Rose     |          4580 |       17816
 Ron      |          4626 |       19699
 Kevin    |          4891 |       17865
 Tod      |          5931 |       17714
\end{verbatim}
\caption*{User Sessions Table}
\caption[User Sessions]{This table shows start and end event counters for different user sessions. Start and end times are omitted for simplicity.}
\label{table:user-sessions}
\end{table}

\begin{lstlisting}[language=Java]
String user_session_actions_query = ``select app_op from events where app_op like 'mint-user-\%' and to_array(app_args)[0] = ? and event_counter > ? and event_counter < ?'';

system.addQuery(``user-sessions-actions'', user_sessions_actions_query);
\end{lstlisting}

Now that we've identified one of Bob's sessions, i.e. the first one, and have a query to find out his actions during that session, we can go ahead and summarize Bob's actions during his first session. For each session we produce a summary that contains the username as well as the actions carried out, and the event counters and timestamps that identify the start and end of that session.

In this example, and the ones that follow, we assume that the method \emph{process(events)} is a user-defined method that  processes a set of events and produces a string summary of those events. In this example, such a summary string would be the comma-delimited list of activites requested by the user in this session. For example, if a user registers Bank of America to their account and downloads all of their transactions in one session, the \emph{process()} method will produce the following summary for that session ``addBank-bankofamerica,downloadTransactions''.

\begin{lstlisting}[language=Java]

Object[] vals = new Object[] {username, start_counter, end_counter, start_time, end_time};
SummaryObject so = new SummaryObject(``user-session-actions'', vals);

String summary = process(so.getEvents());
so.addSummary(``user-session-summary'', String.format(``\%s, \%d, \%d, \%s, \%s'', username, start_counter, end_counter, start_time, end_time, summary));

\end{lstlisting}

\noindent
Finally, we produced a summary for Bob's actions for his first session. This summary contains the username, start and end timestamps and event counters of the session, as well as all the user actions that took place in that session.

\section{Summarizing File System Events}
\label{sum:fs}
In this example we explore how to summarize information about file system events; we show how to store information about file accesses in the last week\footnote{The \emph{DATE\_SUB} substracts an interval of time from a date, as described in \cite{pgsql-gen}.}.

\begin{lstlisting}[language=Java]
String fs_access = ``select * from events where filename<>'' and DATE_SUB(?, INTERVAL 7 DAY)'';

system.addQuery(``fs_week_access'', fs_access);

long now = System.currentTimeMillis();
Object[] vals = Object[]{new java.sql.Timestamp(now)};
SummaryObject so = new SummaryObject(``fs_week_access'', vals);

String summary = process(so.getEvents());
so.addSummary(``File accesses'', String.format(``\%d, \%s'', now, summary));
\end{lstlisting}

\noindent
This will store a summary of the file system events in the last week. Only file system events use the \emph{filename} attribute in the LOG, hence we use that to retrieve all file system events.

After producing the summary, we might wish to mark most file system events, except for the ones that change the file system hierarchy (adding or removing directories or files). We could do this as follows:

\begin{lstlisting}[language=Java]
String fs_protect = ``select * from events where op_name = ? or op_name = ?''
system.addQuery(``protect-fs-events'', fs_protect);
so.mark(``protect-fs-events'', new Object[]{``CREATE_DIRECTORY'', ``REMOVE_DIRECTORY'', ``CREATE_FILE, ``REMOVE_FILE''});
\end{lstlisting}

\section{Summarizing User Trends}

As a final example, we will show how records in the SUMMARY table itself could be summarized. In this section we will further summarize session summaries. To build on what we encountered in section \ref{sum:sessions}, let's assume that more of Bob's sessions have been summarized. We can now coalesce them to detail Bob's activity per week. The following code shows how to do so for the last week.

\begin{lstlisting}[language=Java]

String session_summary_query = ``select * from SUMMARY where name=user-session-summary and to_array(args)[0]=? to_array(args)[3]::timestamp > DATE_SUB(?, INTERVAL 7 DAY) and to_array(args)[3]::timestamp < DATE_SUB(?, INTERVAL 14 DAY)';

system.addQuery(``session-summary'', session_summary_query);
Timestamp now = new Timestamp(System.currentTimeMillis());
Object[] vals = new Object[]{``Bob'', now};
SummaryObject so = new SummaryObject(``session-summary'', vals);

// Create a summary based on the sessions in the last week
String summary = process(so.getEvents());
so.addSummary(``weekly-user-activity'', summary);
\end{lstlisting}


\section{Sufficiently Summarized Information}

The summaries produced in sections \ref{sum:sessions} and \ref{sum:fs} overlap in what they summarize. For example, adding a \emph{user-bank-creds} file to the file system happens only when a user adds a bank to their account, i.e. when they request the \emph{addBank(...)} user action. Hence this particular piece of information, that a user 'A' registered bank 'B' to their account, could be concluded from two summaries: one that summarizes the user's actions per session (or per week), and one summarizing the file system operations.

There are many other examples in which summaries could overlap in what they summarize, and it is one of the reasons why we choose to leave it up to the application to decide if and when certain events are ``sufficiently summarized'', and are ready for marking, rather than trying to determine that ourselves.
