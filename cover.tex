% -*-latex-*-
% 
% For questions, comments, concerns or complaints:
% thesis@mit.edu
% 
%
% $Log: cover.tex,v $
% Revision 1.8  2008/05/13 15:02:15  jdreed
% Degree month is June, not May.  Added note about prevdegrees.
% Arthur Smith's title updated
%
% Revision 1.7  2001/02/08 18:53:16  boojum
% changed some \newpages to \cleardoublepages
%
% Revision 1.6  1999/10/21 14:49:31  boojum
% changed comment referring to documentstyle
%
% Revision 1.5  1999/10/21 14:39:04  boojum
% *** empty log message ***
%
% Revision 1.4  1997/04/18  17:54:10  othomas
% added page numbers on abstract and cover, and made 1 abstract
% page the default rather than 2.  (anne hunter tells me this
% is the new institute standard.)
%
% Revision 1.4  1997/04/18  17:54:10  othomas
% added page numbers on abstract and cover, and made 1 abstract
% page the default rather than 2.  (anne hunter tells me this
% is the new institute standard.)
%
% Revision 1.3  93/05/17  17:06:29  starflt
% Added acknowledgements section (suggested by tompalka)
% 
% Revision 1.2  92/04/22  13:13:13  epeisach
% Fixes for 1991 course 6 requirements
% Phrase "and to grant others the right to do so" has been added to 
% permission clause
% Second copy of abstract is not counted as separate pages so numbering works
% out
% 
% Revision 1.1  92/04/22  13:08:20  epeisach

% NOTE:
% These templates make an effort to conform to the MIT Thesis specifications,
% however the specifications can change.  We recommend that you verify the
% layout of your title page with your thesis advisor and/or the MIT 
% Libraries before printing your final copy.
\title{Summarizing Audit Trails in the Aeolus Security Platform}

\author{Wissam Jarjoui}
% If you wish to list your previous degrees on the cover page, use the 
% previous degrees command:
%       \prevdegrees{A.A., Harvard University (1985)}
% You can use the \\ command to list multiple previous degrees
%       \prevdegrees{B.S., University of California (1978) \\
%                    S.M., Massachusetts Institute of Technology (1981)}
\prevdegrees{S.B., C.S M.I.T., 2011}
\department{Department of Electrical Engineering and Computer Science}

% If the thesis is for two degrees simultaneously, list them both
% separated by \and like this:
% \degree{Doctor of Philosophy \and Master of Science}
\degree{Master of Engineering in Electrical Engineering and Computer Science}

% As of the 2007-08 academic year, valid degree months are September, 
% February, or June.  The default is June.
\degreemonth{September}
\degreeyear{2012}
\thesisdate{August 15, 2012}

%% By default, the thesis will be copyrighted to MIT.  If you need to copyright
%% the thesis to yourself, just specify the `vi' documentclass option.  If for
%% some reason you want to exactly specify the copyright notice text, you can
%% use the \copyrightnoticetext command.  
%\copyrightnoticetext{\copyright IBM, 1990.  Do not open till Xmas.}

% If there is more than one supervisor, use the \supervisor command
% once for each.
\supervisor{Barbara H. Liskov}{Institute Professor}

% This is the department committee chairman, not the thesis committee
% chairman.  You should replace this with your Department's Committee
% Chairman.
\chairman{Christopher J. Terman}{Chairman, Masters of Engineering Thesis Committee}

% Make the titlepage based on the above information.  If you need
% something special and can't use the standard form, you can specify
% the exact text of the titlepage yourself.  Put it in a titlepage
% environment and leave blank lines where you want vertical space.
% The spaces will be adjusted to fill the entire page.  The dotted
% lines for the signatures are made with the \signature command.
\maketitle


% The abstractpage environment sets up everything on the page except
% the text itself.  The title and other header material are put at the
% top of the page, and the supervisors are listed at the bottom.  A
% new page is begun both before and after.  Of course, an abstract may
% be more than one page itself.  If you need more control over the
% format of the page, you can use the abstract environment, which puts
% the word "Abstract" at the beginning and single spaces its text.

%% You can either \input (*not* \include) your abstract file, or you can put
%% the text of the abstract directly between the \begin{abstractpage} and
%% \end{abstractpage} commands.

% First copy: start a new page, and save the page number.
\cleardoublepage
% Uncomment the next line if you do NOT want a page number on your
% abstract and acknowledgments pages.
% \pagestyle{empty}

\setcounter{savepage}{\thepage}
\begin{abstractpage}
% $Log: abstract.tex,v $
% Revision 1.1  93/05/14  14:56:25  starflt
% Initial revision
% 
% Revision 1.1  90/05/04  10:41:01  lwvanels
% Initial revision
% 
%
%% The text of your abstract and nothing else (other than comments) goes here.
%% It will be single-spaced and the rest of the text that is supposed to go on
%% the abstract page will be generated by the abstractpage environment.  This
%% file should be \input (not \include 'd) from cover.tex.
Aeolus is a programming platform that supports the development of secure applications that preserve the confidentiality of information entrusted to them. An important part of the Aeolus platform is an auditing subsystem that maintains a log in which it stores information about every security related event that occurs while applications run. The log allows later analysis to determine whether the security policies of the application have been followed.

For an Aeolus user, analyzing an Aeolus event log can prove to be a daunting task, especially when this log grows to include millions of records. Similarly, storing such an event log can be very costly. The system I present in this thesis provides an interface that allows the creation of user-defined summaries of the Aeolus audit trails, as well as marking of events in the log for future archiving or deletion. Our system makes it easier to analyze the Aeolus event log and less costly to store events of interest. This is done through the use of a \emph{QuerySystem} and \emph{SummaryObjects}.
I present the system in the context of a sample application based on the financial management service \emph{www.mint.com}. The system is an extension to the Aeolus library; it is implemented in Java code and uses PostgreSQL 9.0 as its primary database.





\end{abstractpage}

% Additional copy: start a new page, and reset the page number.  This way,
% the second copy of the abstract is not counted as separate pages.
% Uncomment the next 6 lines if you need two copies of the abstract
% page.
% \setcounter{page}{\thesavepage}
% \begin{abstractpage}
% % $Log: abstract.tex,v $
% Revision 1.1  93/05/14  14:56:25  starflt
% Initial revision
% 
% Revision 1.1  90/05/04  10:41:01  lwvanels
% Initial revision
% 
%
%% The text of your abstract and nothing else (other than comments) goes here.
%% It will be single-spaced and the rest of the text that is supposed to go on
%% the abstract page will be generated by the abstractpage environment.  This
%% file should be \input (not \include 'd) from cover.tex.
Aeolus is a programming platform that supports the development of secure applications that preserve the confidentiality of information entrusted to them. An important part of the Aeolus platform is an auditing subsystem that maintains a log in which it stores information about every security related event that occurs while applications run. The log allows later analysis to determine whether the security policies of the application have been followed.

For an Aeolus user, analyzing an Aeolus event log can prove to be a daunting task, especially when this log grows to include millions of records. Similarly, storing such an event log can be very costly. The system I present in this thesis provides an interface that allows the creation of user-defined summaries of the Aeolus audit trails, as well as marking of events in the log for future archiving or deletion. Our system makes it easier to analyze the Aeolus event log and less costly to store events of interest. This is done through the use of a \emph{QuerySystem} and \emph{SummaryObjects}.
I present the system in the context of a sample application based on the financial management service \emph{www.mint.com}. The system is an extension to the Aeolus library; it is implemented in Java code and uses PostgreSQL 9.0 as its primary database.





% \end{abstractpage}

\cleardoublepage

\section*{Acknowledgments}

First and foremost, I would like to thank my advisor, Prof. Barbara Liskov, for the mentorship she gave me during my two years here at PMG. Her guidance helped me navigate and learn a lot about systems security. I am truly grateful for the understanding, patience and time investment she put in while helping me see this project through; I hope that one day I would be able to reflect those traits as a leader.

A big thank you to my colleague, David Schultz, for his tremendous help in this project and my personal development; from suggesting and brainstorming ideas to improve my project, to explaining technical concepts, and being a strong source of knowledge and inspiration and a good role model in general. David, it has been a pleasure working with you.

I would also like to thank James Cowling, Dan Ports and Barzan Mozafari and everyone in 32-G908 for all the fun conversations and useful advice they provided. I am very fortunate to have had the chance to hear and learn from their experiences, and be inspired by them.

As this thesis culminates five years for me at MIT, I would like to take this chance thank my friends and family. My friends, for being their for me, and for offering me help even when I didn't ask for it. Their presence was a home away from home, and their support guided me in times of uncertainty. My family, without whom I would not be where I am today, for their support and encouragement, for looking out for me, and for setting the bar high. In one way or another, I will always have something to learn from them.

Finally, I would like to thank MIT for all what it has given me, in both computer science and real life. I thank Anne Hunter, my advisors, my professors and my TAs for holding my hand when I took on different challenges. I thank the Stata Center, the Green Building, and The Infinite Corridor for being the monuments of my experience here at MIT, and I thank the Charles River, for being the source of calmness in a very busy time. They will all be missed.


%%%%%%%%%%%%%%%%%%%%%%%%%%%%%%%%%%%%%%%%%%%%%%%%%%%%%%%%%%%%%%%%%%%%%%
% -*-latex-*-
