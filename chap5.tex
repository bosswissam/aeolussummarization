\chapter{The Summary System}

This thesis is concerned with providing mechanisms that allow the size of the log to be controlled. A log for a large system (e.g., the Mint application with millions of users) is going to be very big, and the problem gets worse if the system is long-lived. A way to reduce the log is to delete or archive events. Typically, this will be done for events in the past, e.g., events collected last year are less important than recent events. However, there is no way that Aeolus can decide by itself what events are important: this has to be done by the application. A further point is that the application may need to retain some of the information in the events, at least those that are being deleted; again the application has to specify what information should be retained.

For example, if we store the number of declassifies a user's principal carries out per user session, we would require many fewer events to be stored in the log table to formulate a pattern of the user's activity, and, eventually detect suspicious behavior patterns. Thus, we could choose to delete events that have been summarized, without loss of the summarized information.

Our system is designed to support this process. We provide an interface that can be used by application code to create summaries and to mark events for deletion or archiving; from this point forward we refer to these as the \emph{marked} events. Summaries are stored in a \emph{SUMMARY} table and therefore queries can be used to perform discovery on them if desired.

%The Mint study case highlighted the inherent need for a form of summarization. For example, the query in \ref{query:accesses} depends on the \emph{users} table, which details the system information related to each user. The \emph{users} table in itself can be thought of as a summary, and queries such as the one in \ref{query:accesses} depend on that summary for analysis.

%In the example mentioned in the previous chapter, we also notice that analyzing the event log can become complicated quickly. For example, query in \ref{query:accesses}, whereas it is fairly readable, it shows that the developer would have to write long SQL queries for different forms of analysis, and that there is no intermediate level of abstraction between what the queries read and what they produce.

\section{Summarization Model}

An application uses our summary system from within a VN. This has the following ramifications:

\begin{itemize}
  \item Every access to the LOG or SUMMARY table must obey our information flow constraints. The thread can only read events from these tables whose secrecy labels are contained in those of the thread, and vice versa for integrity labels. Also, all events that are added to the SUMMARY table will have the labels of the thread at the moment the event is added.

The thread producing the summaries or deletes runs with the authority of some principal. For example, a thread creating a summary in the Mint application would likely run the \emph{P$_{MINT}$} principal. Thus the thread can use its authority to declassify as needed to produce a summary with the appropriate labels.
  \item The user code can run an arbitrary Java computation to produce the summary from the events being summarized. This computation could be a simple one, such as a count of the number of the DECLASSIFY events encapsulated by a summary, or a more complex one, such as a categorization of the users activity as normal or suspicious based on the different events encapsulated by the summary.
  \item The activities of the user code will be audited in the usual way. Thus it will be possible to run discovery over these events, and if desired the user code can log additional events in the base log by using the Aeolus interface.
\end{itemize}

\noindent
In the following section, we describe the typical use cases that our system aims to address.

\section{Summarization Workflow}

A typical workflow of how an application might use our system to produce a summary is as follows:

\begin{enumerate}
  \item Possibly run some queries to establish a context for the events to be summarized or marked. For example, identifying Jack's principal ID is necessary to summarize the number of declassifications carried out Jack's principal in a particular session. Such a context could be established by retrieving a \emph{Users Table} as shown in section \ref{audit:mint-example}\footnote{Other necessary information for this summary example is to identify the start and end of Jack's sessions.}.
  \item Run a query to identify the events to be summarized or marked, for example, all the declassifications carried out by Jack's principal per session.
  \item Compute the information that goes into the summary, for example, whether or not the number of declassifications caused by Jack's principal for a particular session is too high.
  \item Possibly produce and store the summary.
  \item Possibly mark all events covered by the summary.
\end{enumerate}

\noindent
The next two sections describe how our system allows applications to query, summarize and mark events.

\section{The Query System}

Steps 1 and 2 above are used to establish the context and then to identify the events to be summarized or marked. Since these activities are likely to be performed periodically, it is useful to allow the queries to be defined once and then used over and over.


For example, the query \ref{code:query-accesses} could be run many times over the lifetime of the application.

However, an application might need to use a particular query with different parameters - for example, the Mint application might want to run a query to find out the last login time for different users.

A further point is that queries could be run over the LOG or the SUMMARY table, or both. It is up to the application to specifically identify which tables the query will run over.

The following two sections describe the interface we provide to support these use cases, as well as the \emph{QUERY} table which we use to store query records.

\subsection{The Interface}
\label{model:query-system}

An application can use our system by calling \lstinline$QuerySystem.getInstance()$. This will provide the application with an instance of our system, which can then be used to call the following methods for managing and running queries in the QUERY table:
\begin{description}
  \item[\emph{addQuery(name, query\_text)}] \ \\
    Adds a query with a NAME as \emph{name}
    and TEXT as \emph{query\_text}. If a query with that
    name already exists, throws an exception.
  \item[\emph{runQuery(name, vals)}] \ \\
    Runs the query with the name \emph{name}, 
    replacing any existing
    parameter placeholders in the query TEXT with
    the values in the \emph{vals} array, and 
    returns the underlying set of event records.
\end{description}

\noindent
The \emph{addQuery} and \emph{runQuery} methods log the type of the operation, e.g. ``query-add'' or ``query-run'', as well as the name (and query parameters in the case of \emph{runQuery}) of the query involved.

We do not support deletion of queries from the QUERY table because summaries could refer to them, as described in section \ref{model:summaries}.

A final point is that \emph{runQuery} will return only records that the calling thread is allowed to see per the rules in section \ref{difc:rules}.

\subsection{Query Attributes}

For each record stored in the QUERY table, we store the following attributes:

\begin{description}
  \item[\emph{NAME}] \ \\
    This field is provided by the application. It
    can be used to identify the purpose of the query,
    for example, query \ref{code:mint-signup}
    could be named ``Mint Sign-ups''.
  \item[\emph{TEXT}] \ \\
    This is the text of the query. This does
    not have to be a complete query: we allow the
    application to leave query parameters unspecified,
    replaced with placeholders in this field,
    so that a record can be re-used with different
    parameters to represent different queries.
    The following is an example of what this
    attribute might hold as a value:

\begin{lstlisting}[language=SQL, deletendkeywords={TIMESTAMP}]
select * from LOG where event_counter=?
\end{lstlisting}

\end{description}

\noindent
We do not store labels in the QUERY table because we expect those queries to run over the LOG and SUMMARY tables, which already have labels. To avoid any confusion and prevent the use of our interface as a covert channel, we require that a thread have null labels when adding a query to the QUERY table.

\section{Producing Summaries and Marking Events}
\label{model:summaries}
An application can use queries in the QUERY table to set up a context for summarization and to identify events to be summarized. The application can then store a summary of those events in the SUMMARY table. The application might then choose to mark the events encapsulated by a particular summary event, or summarize a group of summary events.

For example, a summary of the information presented in figure \ref{table:bob-declass} would detail all the users other than Bob who accessed Bob's information, along with the period of time during which those accesses happened (identified by a start event\_counter and an end event\_counter).

The following two sections describe the interface we provide to support producing summaries and marking events, as well as the SUMMARY table which we use to store summary records.

\subsection{SummaryObjects}
\label{model:summary-objects}

In this section we describe \emph{SummaryObjects}. \emph{SummaryObjects} provide an interface for applications to create and add summaries to the SUMMARY table, as well as mark their events.

The inteface we expose is as follows:
\begin{description}
  \item[\emph{SummaryObject(query\_name, query\_params)}] \ \\
    Retrieves the events identified by the query
    \emph{query\_name} with parameters \emph{query\_params}.
  \item[\emph{addSummary(summary\_name, summary\_args)}] \ \\
    Adds the summary represented by this 
    \emph{SummaryObject} to the SUMMARY table with
    name as \emph{summary\_name} and arguments
    as \emph{summary\_args}.
    Stores the calling threads' secrecy and integrity
    labels of the summary object.
  \item[\emph{mark(query\_name, query\_params)}] \ \\
    Marks the group of events underlying this
    \emph{SummaryObject}. The \emph{query\_name}
    and \emph{query\_params} identify a subset
    of the events covered by this summary object
    that should NOT be marked.
    If the arguments are \emph{null}, all events
    covered by this summary object will be marked.
\end{description}

\noindent
The reason we allow some events to be identified as not markable is because there are events in the log that Aeolus won't allow to be deleted; in particular all events that record changes to the authority state, e.g., creating a principal, or allowing one principal ot act-for another. It seems likely that the application will also have some events that it does not allow to be deleted. The arguments to the mark method allow these events to be identified.

In our scheme, we depend on being able to identify events underlying a particular summary by their event\_counter. Hence we require all queries used to produce \emph{SummaryObjects} to include an \emph{event\_counter} attribute.

Each use of a \emph{SummaryObject} method is logged to the LOG table; this does not apply to the constructor. We log the name of the function along with the arguments to identify the exact actions of the application.

Finally, The \emph{addSummary} and \emph{mark} methods log the type of the operation, e.g. ``summary-add'' or ``summary-mark'', as well as the name and parameters of the underlying query.

\subsection{Summary Attributes}

For each record stored in the SUMMARY table, we store the following attributes:

\begin{description}
  \item[\emph{EVENT\_COUNTER}] \ \\
    Unique identifier for each record.
  \item[\emph{NAME}] \ \\
    Similar to \emph{NAME} in the QUERY table,
    this field is provided by the application and
    could be used to identify the content of the
    summary record.
  \item[\emph{ARGS}] \ \\
    This field is provided by the application.
    It stores the application's computation of the summary.
    In the example mentioned above, this could be the 
    list of users accessing Bob's information over the
    period of time elapsed between the event
    identified by \emph{START\_COUNTER} and the
    event identified by \emph{END\_COUNTER}.
  \item[\emph{QUERY\_NAME}] \ \\
    This field is provided by the application.
    It identifies the query in the QUERY table
    that identifies the events summarized
    by this record.
  \item[\emph{QUERY\_PARAMS}] \ \\
    This field is provided by the application.
    It stores the parameter values for the query
    identified by the \emph{QUERY\_ID} if any.
  \item[\emph{SECRECY}] \ \\
    This is the secrecy label of the thread 
    that added the summary record.
  \item[\emph{INTEGRITY}] \ \\
    This is the integrity label of the thread 
    that added the summary record.
  \item[\emph{TIMESTAMP}] \ \\
    The real time at which this summary was created.
\end{description}

\section{Marking Events}

Our design allows but does not require the production of a summary event in conjuction with marking. For instance, a \emph{SummaryObject} can be created, then used to mark events, without the application having to call \emph{addSummary}.

Since our system was not intended to alter the LOG or SUMMARY table, we keep track of marked events in two additional tables: LOG-MARKED and SUMMARY-MARKED. Those tables contain event counters of events that have been marked by the application.

Our system does not show marked events in the results of later queries. Since those events are not deleted, they and could be made visible to the user, although our system does not do this at present.

However our system only provides a mechanism to identify marked events, and leaves the question of how to archive and delete those events as further work.

%The following sections describe how our summarization system allows developers to accomplish three things: use a clear interface to query the event log, reason more easily about events in the past, and improve the performance of their system in terms of runtime and storage costs.

%\subsection{Versioning}

%abstraction layer to treat summaries like base events?

\section{Discussion}

It is possible through our system that an application summarizes pre-existing summaries. For example, the Mint application, after summarizing declassify events for user data tags per user session, could use those summaries to summarize declassifiy events for user data tags per month.

Our system supports this functionality since it does not make any strong assumption about queries in the QUERY table. This allows the application to treat events in the LOG or SUMMARY table the same way.

Finally, by allowing applications to protect events from being marked, and internally preventing important Aeolus events from being marked, our system provides a way to prevent application errors from deleting important information.
