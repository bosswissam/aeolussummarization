\chapter{Summary of Contributions and Future Work}

\section{Contributions}

In this thesis, we present a summarization system that allows for applications to summarize information in their Aeolus audit trails, as well as mark events for later archival or deletion. We also present a query system that makes it easier for applications to both use our summarization system and query the Aeolus audit trails in general. Finally, with the use of summarization, information could span much less events in summarized form than in raw form, allowing the application to write faster queries and save space.

\section{Future Work}

The are two main ways in which our system could be extended in the future. The first is to give applications more power through the query system. It is unclear how much power exactly should be given to the application, however, it is clear that applications should be able to do more than just select events from the LOG and SUMMARY tables through their queries, for example, in the Mint application, it would be really useful if the application could define a view that would produce the signup information of all users (shown in table \ref{table:users-info} whenever requested. This would be an improvement upon the current of the system because the application wont have to store this information in memory, rather it is stored in the database and can be access again in the future without being recomputed. Our system could benefit from future.

The second possible way to extend our system is to investigate how archival and deletion sholud be carried out, and build an interface that allows applications to do so. Such an interface would perhaps provide some flexibility to the application, while still providing some working framework for the application to use. This interface would have to be clear about how it operates, lest the application misuses it and important information is lost. Such an interface might also include features that build on our work, such as allowing the application to specify sets of events, through queries, that should never be deleted or archived.
